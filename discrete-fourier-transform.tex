\documentclass[t,xcolor={svgnames},aspectratio=169]{beamer}
\usetheme{metropolis}

\title{Briefly introducing the discrete Fourier transform}
\author{Dorotea Avra}

\begin{document}

\maketitle

\begin{frame}{Background}
    \begin{itemize}
        \item From Wikipedia:\footnote{\url{https://en.wikipedia.org/wiki/Discrete_Fourier_transform}}
    \end{itemize}
    \begin{quotation}
        the discrete Fourier transform is a discrete version of the Fourier
        transform that converts a finite sequence of equally-spaced samples of a
        function to a same-length sequence of equally spaced samples of the
        discrete-time Fourier transform [\ldots].
    \end{quotation}
\end{frame}

\begin{frame}{Background}
    \begin{itemize}
        \item If data sets contain periodic oscillations, one can analyse
            them with ``spectral methods'', a.k.a. ``Fourier transform
            methods''.
        \item Sometimes it is easier or more useful to analyse and process
            data in the frequency domain than it is in the time domain.
        \item The discrete Fourier transform is a numerical method to
            calculate the Fourier transform of a data set on a computer.
    \end{itemize}
\end{frame}

\begin{frame}{Discrete Fourier transformation definition}
    Consider a vector of $N$ evenly-spaced time series
    points,\footnote{Discussion based on Chapter 5.2 Spectral Analysis from
    \emph{Numerical Methods for Physics} by Alejandro L.\ Garcia, 1994.}

    \begin{equation}
        \mathbf{y} = [y_1, y_2, \ldots, y_N]
    \end{equation}

    The data are sampled every $\tau$ seconds, thus giving us a list of time
    points defined by:

    \begin{equation}
        t_i = \tau (i - 1)
    \end{equation}

    where $i$ is the 1-based index of the input vector.
\end{frame}

\begin{frame}{Discrete Fourier transformation definition}
    Given the vector of data points, $\mathbf{y}$, we can define its discrete
    Fourier transform, $\mathbf{Y}$, as a vector:

    \begin{equation}
        Y_{k+1} = \sum_{j=0}^{N-1} y_{j+1} e^{-2 \pi i j k / N}
    \end{equation}

    where $i = \sqrt{-1}$.  Note that $j$ and $k$ start at zero, whereas the
    vector uses a 1-based index.
\end{frame}

\begin{frame}{Example}
    Following the discrete Fourier transform discussion in
    Garcia,\footnote{Chapter 5.2 Spectral Analysis from \emph{Numerical Methods for
    Physics} by Alejandro L.\ Garcia, 1994.} using a sampling interval of
    $\tau = 1$, $N = 50$ data points, a signal frequency of $f_s = 0.2$, and
    a phase of $\phi_s = 0$ we obtain the following graph:

    \centerline{%
        \includegraphics[width=0.5\textwidth]{dft-sine-wave.png}
    }
\end{frame}

\end{document}
